\documentclass[10pt]{letter}

\usepackage[margin=2.54cm, a4paper]{geometry}
\usepackage[utf8]{inputenc}
\usepackage{lscape}
\usepackage{longtable}
\usepackage{hhline}
\usepackage{booktabs}
\usepackage{xcolor}
%%\usepackage[final]{changes}
\begin{document}

\underline{Editor's comments} \\
\\
\textit{
After considering the comments of the two reviewers, which are provided below,
I think your manuscript should be acceptable for publication after satisfactory
revision.  Please address the reviewers' comments when revising your
manuscript. Thank you again for submitting your manuscript to Ecological
Modelling.
} 

Thank you for your comments, we have provided a detailed response to each of
the points made by the reviewers along with a track-changed (as well as a
`clean') revision of the manuscript. In addition, we provide a short vignette
detailing an example application and R code for the framework, which we hope
will widen accessibility and take up of the modelling framework. We hope that
together this satisfies your requirements and look forward to seeing the
manuscript published. \\


\underline{Reviewer 1 comments} \\

\textit{This paper introduces a simulation framework for assessing the effects
	of different spatial and temporal observation scales on management of
	mixed fisheries. An application of the framework is presented, and the
	findings from the example application provided insights into the
	importance of data resolution on design and effectiveness of closed
	areas. The model code is included as an R package so that other
	researchers will be able to implement it readily to address a wide
	variety of management questions. The question of appropriate scales is
	an important one, and I think this will be a valuable tool for
	researchers and managers. I recommend that it be published after minor
	revisions.} 

Thank you, we are grateful for your comments that have improved our manuscript
and look forward to our work reaching and being used by a broad audience. 


	\begin{center}
	\begin{longtable}{p{8cm} | p{8cm}}
%	\caption{Highly resolved spatiotemporal simulations for exploring mixed
%	fishery dynamics}
		\toprule
		Comment & Response \\
		\\
		\hline
		\\
\textbf{Asbstract} & \\
\\
I think it would be helpful to mention the name of the R package (MixFishSim)
that accompanies this paper somewhere in the abstract. People will search for
the package name, it would be nice if the term was included in the abstract so
the paper will show up in the search too. & We have included
\textit{MixFishSim} in the first sentence of the abstract. \\
\\
``true underlying populations" - somehow make sure readers know this is
referring to simulated populations. & We have changed to "true (simulated)
underlying populations" on line 10 of the abstract. \\
\\
Could add something like "bycatch avoidance", or something like that, somewhere
in key words or abstract; people searching for that term would be glad to find
this paper, I would think! & Done, see keywords. \\ 
\\
I think the abstract could be streamlined a bit. You seem to have 2 major
objectives with this paper: 1) introducing and describing the simulation
framework that could answer a number of important management questions, and 2)
demonstrating its usefulness with an example application. One full paragraph
for each should be adequate, with a conclusion that states the usefulness of
the framework for answering a number of fisheries management questions.  &
We've worked to shorten and strengthen the abstract as suggested. \\
\\
\hline
\\
\textbf{Introduction} &  \\
\\
Line 62: ``Do different sources of sampling-derived fisheries data reflect"… &
Corrected at line 73. \\
\\
		
		\textbf{Methods} &  \\
\\
Why was the Matern covariance structure chosen? I am just curious about this;
does it reflect the spatial autocorrelation of animal distributions better than
a Gaussian or other type of covariance structure? & We used the Matérn
covariance structure as it is flexible, and contains the exponential form as a
special case.  We've added text at line 170 to say ``We use the most commonly
used Matérn covariance structure as it is a flexible form that under certain
conditions is of the same form as an exponential function and it enables us to
model the spatial autocorrelation observed in animal populations where density
is more similar in nearby locations, but that correlation decreases
non-linearly...." \\ 
\\
It would be helpful to have explanation of thermal tolerance earlier than line
155 & We have now added a sentence at the beginning of the section (lines 153 -
154) that reads ``described by a set of probabilities.  Stochastic
probabilities are affected by the suitability of habitat, temperature in a cell
and the thermal tolerance of a population to that temperature." \\
\\
The paragraph starting at line 179 would be helpful to have at the beginning of
the section & Agree, we have moved this to now be the second paragraph in the
section. \\
\\
Line 187, make tenses consistent (e.g. determined vs. determines) - also this
sentence is a little awkward to read. & We have changed lines 208 - 218
so that it reads in the past tense and rewritten the paragraph to improve
readability. \\
\\
\hline
\\
\textbf{Results} &  \\
\\
Line 376: What does ``Visualized using Gerritsen (2014)" mean? & We've now
added to the caption of Current Figure 3: ``The figure shows catch composition
at each spatial unit represented by a square pie chart of the four populations.
The area of each colour is proportional to the weight of each population caught
in that unit. Figure produced with the R package `mapplots' (Gerritsen et al
2014)." \\ 
\\
Line 397: Are these the same figures in Table 7? If so, just refer to the table
and summarize & Yes, they are the same and we agree this change improves
readability and the two final paragraphs in section 4.2 (lines 424 - 432) have
been revised accordingly. \\ 
\\
Line 400: move \% sign & This paragraph has been revised so the point
	is redundant. \\
\\
Line 406: ``(in red)" - leave reference to color for figure legend & Removed
reference to color on line 435. \\
\\
Line 410: ``Again" doesn't need to be there, same in line 413 & Agree, it has
been removed on line 430. \\
\\
Line 416: This section is a little confusing and needs to be
refined/streamlined. & We have now revised this section to clarify and improve
readability. For example, shortening sentences and ensuring one point is made
per sentence (lines 443 - 455). \\
\\
Line 418: This paragraph should be in methods. But you did it for all
populations, right? Not just population 3? & That's correct, the closures
affected all populations but the closures were targeted to reduce fishing
mortality on population 3. We have clarified and moved to the methods section
now at lines 378 - 386.
\\
\\
\hline
\\
\textbf{Discussion / Conclusion} &  \\
\\
The discussion and conclusion overlap a bit; I felt I was reading similar
things over again; perhaps these can be streamlined? & We have reviewed but
believe each paragraph makes an independent point except the final paragraph,
which does repeat some of the previous conclusions. As such, we have removed
the final paragraph of the conclusions. \\
\\
Line 560: change ``reduced" to ``reduce" & Done on line 581. \\
\\
Line 663: change ``hypothesis" to ``hypotheses" & This paragraph has now been
removed at the suggestion of reviewer 2. \\
\\
\hline
\\
\textbf{Figures} &  \\
\\
Figure 1: nicely summarizes the model. It took me a little bit to figure out
that ``Rec" referred to recruitment - even though it was in the legend (I
missed it when I read it the first time)… is it possible to choose an
abbreviation that makes it more obvious, maybe something like ``Recruit" or
even ``Recr"? It also occurs to me that there is an opportunity to introduce
the terminology from your R package here to make it very clear to the people
using your code how the functions work together… this is only an idea though…
or maybe a similar figure could be included with the vignette. & We have
changed to `Recr' to make clearer. Rather than change the schematic, we have
included a short example vignette demonstrating the use of the code to increase
accessibility and further uptake. \\
\\
Figure 5: Do the colors indicate the dominant population in each cell? It is
hard to imagine that one would assume only the presence of specific population
in the cell? Or do I have the wrong idea about what the plot is showing? & Each
cell has the four colours in proportion of the weight of each population caught
(as a square pie chart) - we've clarified this in the figure caption. \\
\\
Figure 6: Add what the different colors mean to the legend & This is already
included as the legend shows which population is represented by which colour. \\
\\
Figure 9: So the top two panels show spatial distribution for population 3?
That needs to be added to the legend. Also there should be a red box on the top
plot too. & The top two panels show the spatial distribution of the fishing
effort before (a) and after (b) the spatial closures, while the bottom (c)
shows the suitable habitat for the population. We've clarified in the figure
caption. \\ 
\bottomrule
	\end{longtable}

\end{center}

\underline{Reviewer 2 comments} \\

\textit{The paper Dolder et al. presents a new highly resolved spatiotemporal
	model that simulates populations of species and fishery dynamics to
	assess if data from commercial catches and fixed-site sampling surveys
	on different spatiotemporal scales represent the real distribution of
	the populations, and if these data are effective to base  management
	programs. Moreover, they presents other applications of the model.} \\

\textit{This is a very interesting paper, and the model provides several
	applications for management programs. I have only few greater questions
	and some small issues, most of them about the presentation of the
	information.}  \\

We thank the reviewer for their comments and suggestions that have greatly
improved the presentation and interpretation of our methods and results.

\begin{center}


	\begin{longtable}{p{8cm} | p{8cm}}
%	\caption{Highly resolved spatiotemporal simulations for exploring mixed
%	fishery dynamics}
		\toprule
		Comment & Response \\
\\
\hline
\\	
	\textbf{General comments} &  \\
\\
		 How many species does the model simulate? The authors refer to
		 the issue of catching unwanted species (vulnerable or low
		 quota species) when available quota species is caught, and it
		 suggested me that populations of different species is modelled
		 at the same time. However, they  did not describe how many
		 species could be simulated in each simulation, only that four
		 populations were simulated. Does the populations simulated in
		 different simulations, or all of them are simulated
		 simultaneously? The results indicate that the four populations
		 were simulated in the same simulation. Please, clarify this in
		 the MM.  & We now emphasise on line 100 that the framework
		 allows the user to define as many species as required. There
		 will be some computational/memory cap but we see no reasons
		 why you could not include 10s or more. We chose four to keep
		 the analysis more tractable. All the populations are part of
		 the same system, irrespective of the number of species
		 included, hence they are simulated in the same simulation.\\
\\
There are too many figures and they should be better edited. Why the name of
the row and columns are inside the graphics? The axes scales and the axes title
should be greater. I would also send the figure 2, 3 and 9 as supplementary
data.  & We have revised the main figures to increase text size and changed to
bold font; we have also moved figures 2 and 3 to the supplementary figures. We
have retained figure 9 (Current Figure 7) as we believe it gives understanding
to the fishery - population interactions before and after the spatial closures
which is important for the narrative of the example application. \\
\\
The legends of the supplementary figures should be more informative. In the
Figure S2, what the light gray spots are? Habitat preference? Are the squares
the spawning areas? In the figure S3, are each square a week?  & We have
revised the captions for the supplementary figures to provide additional
information, including the issues you've identified. Specifically, Figure S2
(Current Figure S3): ``with the darker colour showing greater habitat
suitability". Figure S3 (Current Figure S7): the caption reads ``for 52 separate
weeks"

\\
I am not native English, but the language seems good for me. However, I missed
several commas along the text that make the sentences too long and hard to
understand. I suggest a revision of this.  & We have revised the text to ensure
readability. \\

\hline
\\
\textbf{Small issues} &  \\
\\
Line 14 - 18: This phrase is too long and hard to understand. Please, rephrase
this in shorter sentences to clarify this idea. & These sentences have been
revised to clarify in lines 14 - 21.  \\
\\
Lines 20 - 43: These two paragraphs seem redundant. & We believe these
paragraphs set the context for an important premise of the paper: That
identifying the correct spatial scale for management is important to achieve
the anticipated outcomes; particularly in mixed fisheries where there is a
`cost' through loss of fishing opportunities for other fisheries which
management seeks to minimise. However, we have revised at lines 23 - 34 to make
our point clearer to the reader. \\
\\
Line 68 - 72: The authors mentioned that the MixFishSim could be used to infer
if fisheries data are the real community structure. I suggest adding one or two
phrases to explain how the model could help in this. & We explain that as our
community structure is simulated and therefore known, by comparing our inferred
community structure from commercial catches with our known true structure we
can understand how biases introduced by using commercial fishing as a sample
affects our understanding. These points are covered on lines 79 - 83.\\
\\
Line 90. The authors presented the time-step of each module, except the
Recruitment dynamics. The recruitment dynamics probably follow the population
dynamic, but it would be interesting present this explicitly. & Recruitment
occurs along with other population dynamics, but only during defined periods
for each population. We've clarified by saying in lines 100 - 102 ``Population
dynamics for any number of species, as chosen by the user, operate on a daily
time-step \textit{(with recruitment occurring only during defined seasons for
	each population),} while.."
\\
\\
Line 282: How these population parameters were selected? Randomly? Using real
data? Or the authors only created them? & The demographic parameters were
chosen by the authors to broadly reflect population dynamics for a four-species
assemblage representing species typically found in a demersal fishery. We have
clarified by including in lines 310 - 313 the sentence: ``The population
demographics were chosen to broadly represent three mobile low-medium value
groundfish species and one high value species with low mobility, with the
dynamics hypothetical but as you might expect to find in a typical demersal
fishery." \\
\\ 
Line 426: What does adapted means in this context? Exploring new opened areas?
& Yes, we have clarified in the text on lines 446 - 447 that the
fisheries ``adapted" to the closures textit{by fishing new areas of high
	abundance to fish}\\
\\
Line 613: Do not use parenthesis inside of other parenthesis. & We have changed
on lines 632 from ``( e.g." to ``, e.g." to eliminate one set of parentheses.
\\
\\
Table 2 and table 3 are not cited in the text Citation of the supplementary
figures are out of order & This is now corrected with references inserted for
Table 2 (line 204) and Table 3 (line 274) and the supplementary figures
reordered. \\
\\
Figure 3: Individual years ARE the light grey lines    & Corrected. \\
\\
Figure 4. What do the purple spots represent? Describe these spots in the
capture & This has been clarified (in Current Figure 2) to be the revenue at
each location (darker areas indicate higher revenue). \\ 
\\
Figure 7. Add what ``res"  means in the caption.  & Have clarified in the
legend text that it means ``resolution" \\
\\
Figure 8. What F means? & Changed to read ``Fishing mortality" \\
\\
		\bottomrule
	\end{longtable}

\end{center}

\end{document}
