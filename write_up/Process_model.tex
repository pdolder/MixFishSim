
%%%%%%%%%%%%%%%%%%%%%%%%%%%%%%%%%%%
% Overview schematic  
\begin{figure}[!ht]
	\hspace{-4em}
\begin{tikzpicture}[%
    >=triangle 60,              % Nice arrows; your taste may be different
    start chain=going below,    % General flow is top-to-bottom
    node distance=8mm and 60mm, % Global setup of box spacing
    every join/.style={norm},   % Default linetype for connecting boxes
    scale=0.9,
    every node/.style={scale=0.8}]                 
    \label{fig:model}
    % ------------------------------------------------- 
% A few box styles 
% <on chain> *and* <on grid> reduce the need for manual relative
% positioning of nodes
\tikzset{
  base/.style={draw, on chain, on grid, align=center, minimum height=4ex},
  proc/.style={base, rectangle, text width=8em},
  test/.style={base, rectangle, text width=5em},
  term/.style={proc, rounded corners},
  % coord node style is used for placing corners of connecting lines
  coord/.style={coordinate, on chain, on grid, node distance=6mm and 25mm},
  % nmark node style is used for coordinate debugging marks
  nmark/.style={draw, cyan, circle, font={\sffamily\bfseries}},
  % -------------------------------------------------
  % Connector line styles for different parts of the diagram
  norm/.style={->, draw, lcnorm},
  free/.style={->, draw, lcfree},
  cong/.style={->, draw, lccong},
  it/.style={font={\small\itshape}}
}
% -------------------------------------------------
% Start by placing the nodes in the middle
\node [proc, densely dotted, it] (p0) {start};
% Use join to connect a node to the previous one 
\node [term, join, fill=lcnorm!25]      {Choose random location to fish};
\node [term, join, fill=lcfree!25] (p1){Sample catch};
\node [test, densely dotted] (t1) {if $(Rec)$};
\node [test, join, densely dotted] (t2) {if (Pop Movement)};
\node [test, join, densely dotted] (wk) {if (Pop Dynamics)};
\node [test, join, densely dotted] (t3) {if (Past Knowledge)};
\node [term, join, fill=lcnorm!25]  (p2) {Correlated Random Walk};
\node [term, join, fill=lcnorm!25]  (p3) {Choose new fishing location};
\node [proc, densely dotted, join, it] (p7) {end};

% Right nodes
\node [proc, fill=lcfree!25, right=of p1] (p4) {Popn depletion};
\node [proc, fill=lcfree!25, right=of t1] (p5) {Popn Recruitment};
\node [proc, fill=lcfree!25] (p6) {Popn movement};
\node [proc, fill=lcfree!25, right=of wk] (wk1) {Growth and natural
	mortality};

% left nodes
\node [test, fill=lcnorm!25, left=of t3] (t4) {Use past fishing locations};

% -------------------------------------------------
% A couple of boxes have annotations
\node [below=of p0, it, yshift=1.5em,xshift=-3em] {$t=1$};
\node [right=30mm of p3, it] {$1 < t < t_{max}$};
\node [below=of p3, it,xshift=-3em, yshift=1.5em] {$t = t_{max}$};

% -------------------------------------------------
% First, the straight north-south connections. In each case, we first
% draw a path with a (consistently positioned) annotation node, then
% we draw the arrow itself.
  \draw [*->,lcfree!50] (p1) -- (p4);
\path (t1) to node [near start, xshift=2em, yshift=0.5em] {$y$} (p5); 
  \draw [*->,lcfree!50] (t1) -- (p5);
\path (t2) to node [near start, xshift=2em, yshift=0.5em] {$y$} (p6); 
  \draw [*->,lcfree!50] (t2) -- (p6);
\path (t3) to node [near start, xshift=-3em, yshift=0.5em] {$y$} (t4); 
  \draw [*->,lccong!50] (t3) -- (t4);
\path (wk) to node [near start, xshift=2em, yshift=0.5em] {$y$} (wk1); 
  \draw [*->,lcfree!50] (wk) -- (wk1);

% left to right paths
\path (t1) to node [near start, yshift=-0.2em] {$n$} (t2) ;  
\path (t2) to node [near start, yshift=0.8em] {$n$} (t3) ;  
\path (wk) to node [near start, yshift=-0.2em] {$n$} (t3) ;  
\path (t3) to node [near start, yshift=-0.3em] {$n$} (p2) ;  

% ------------------------------------------------- 
% the twisty connectors. Again, we place the annotation
% first, then draw the connector

\node [coord, left=of p3] (c1)  {};  
\draw [*->,lccong!50] (t4.south) -- (c1) |- (p3);

\node[coord, below=of p4] (c2) {};
\node[coord, above=of t1] (c3) {};
\draw [-<,lcfree!50] (p4.south) -- (c2) |- (c3) |- (t1.north);

\node[coord, below=of p5] (c4) {};
\draw [->,lcfree!50] (p5.south) -- (c4) |- ($(t2.east) + (0,0.5)$);

\node[coord, below=of p6] (c71) {};
\draw [->,lcfree!50] (p6.south) -- (c71) |- ($(wk.east) + (0,0.56)$);

\node[coord, below=of wk1] (c7) {};
\draw [->,lcfree!50] (wk1.south) -- (c7) |- ($(t3.east) + (0,0.3)$);

% -------------------------------------------------
% A last flourish which breaks all the rules
\draw [->,purple, dotted, thick, shorten >=1mm]
  (p3.south) -- ++(5mm,-3mm)  -- ++(87mm,0mm) 
  |- node [black, near end, yshift=1.75em, it]
    {} ($(p1.north) + (0,0.3)$);
% -------------------------------------------------
\end{tikzpicture}
\caption{Overview Schematic of simulation model. The blue boxes indicate fleet
	dynamics processes, the green boxes population dynamics processes while
	the white boxes are the time steps at which processes occur; $t$ = tow,
	$tmax$ is the total number of tows; (Rec), (Pop Movement), (Pop
	Dynamics) logic gates for recruitment periods, population movement and
	population dynamics for each of the populations, (Past Knowledge) a
	switch whether to use a random (exploratory) or past knowledge
	(exploitation) fishing strategy.}
%%\label{fig:overview}
\end{figure}
%%%%%%%%%%%%%%%%%%%%%%%%%%

