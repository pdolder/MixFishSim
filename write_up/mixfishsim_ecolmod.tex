\documentclass[review]{elsarticle}

\usepackage{lineno,hyperref}
\modulolinenumbers[5]

\journal{Ecological Modelling}

%%%%%%%%%%%%%%%%%%%%%%%
%% Elsevier bibliography styles
%%%%%%%%%%%%%%%%%%%%%%%
%% To change the style, put a % in front of the second line of the current style and
%% remove the % from the second line of the style you would like to use.
%%%%%%%%%%%%%%%%%%%%%%%

%% Numbered
%\bibliographystyle{model1-num-names}
%% Numbered without titles
%\bibliographystyle{model1a-num-names}
%% Harvard
%\bibliographystyle{model2-names.bst}\biboptions{authoryear}
%% Vancouver numbered
%\usepackage{numcompress}\bibliographystyle{model3-num-names}
%% Vancouver name/year
%\usepackage{numcompress}\bibliographystyle{model4-names}\biboptions{authoryear}
%% APA style
%\bibliographystyle{model5-names}\biboptions{authoryear}
%% AMA style
%\usepackage{numcompress}\bibliographystyle{model6-num-names}
%% `Elsevier LaTeX' style
\bibliographystyle{elsarticle-num}
%%%%%%%%%%%%%%%%%%%%%%%

\begin{document}

\begin{frontmatter}

\title{A mixed fishery simulation framework}

%% Group authors per affiliation:
\author[1,2]{Paul J. Dolder\corref{c}}
\cortext[c]{Corresponding author}
\ead{paul.dolder@gmit.ie}

\author[1]{Coilin Minto}

\address[1]{Galway-Mayo Institute of Technology (GMIT), Dublin Road, Galway,
	Ireland} 
\address[2]{Centre for Environment, Fisheries and Aquaculture Science (Cefas),
	Pakefield Road, Lowestoft, UK}

\begin{abstract}
A concise and factual abstract is required. The abstract should state briefly
the purpose of the research, the principal results and major conclusions. An
abstract is often presented separately from the article, so it must be able to
stand alone. For this reason, References should be avoided, but if essential,
then cite the author(s) and year(s). Also, non-standard or uncommon
abbreviations should be avoided, but if essential they must be defined at their
first mention in the abstract itself.  
Graphical abstract: Although a graphical abstract is optional, its use is
encouraged as it draws more attention to the online article. The graphical
abstract should summarize the contents of the article in a concise, pictorial
form designed to capture the attention of a wide readership. Graphical
abstracts should be submitted as a separate file in the online submission
system. Image size: Please provide an image with a minimum of 531 × 1328 pixels
(h × w) or proportionally more. The image should be readable at a size of 5 x
13 cm using a regular screen resolution of 96 dpi.  Preferred file types: TIFF,
EPS, PDF or MS Office files.
\end{abstract}

\begin{keyword}
Some\sep keywords \sep here. Max 6, American "spelling" 
\MSC[2010] 00-01\sep  99-00
\end{keyword}

\end{frontmatter}

\linenumbers

\section{Introduction}

State the objectives of the work and provide an adequate background, avoiding a
detailed literature survey or a summary of the results.

\subsection{Can subsection like this}

\section{Materials and Methods}
Provide sufficient details to allow the work to be reproduced by an independent
researcher. Methods that are already published should be summarized, and
indicated by a reference.  If quoting directly from a previously published
method, use quotation marks and also cite the source. Any modifications to
existing methods should also be described.

\section{Theory/calculation}

A Theory section should extend, not repeat, the background to the article
already dealt with in the Introduction and lay the foundation for further work.
In contrast, a Calculation section represents a practical development from a
theoretical basis.

\section{Results}

Results should be clear and concise.

\section{Discussion}

This should explore the significance of the results of the work, not repeat
them. A combined Results
and Discussion section is often appropriate. Avoid extensive citations and
discussion of published
literature.

\section{Conclusions}

The main conclusions of the study may be presented in a short Conclusions
section, which may stand
alone or form a subsection of a Discussion or Results and Discussion section.

\section*{Appendices}

If there is more than one appendix, they should be identified as A, B, etc.
Formulae and equations in appendices should be given separate numbering: Eq.
(A.1), Eq. (A.2), etc.; in a subsequent appendix, Eq. (B.1) and so on.
Similarly for tables and figures: Table A.1; Fig. A.1, etc.

\section*{Abbreviations} Detail any unusual ones used.

\section*{Acknowledgements} those providing help during the research..

\section*{Funding} This work was supported by the MARES doctoral training
program; and the Centre for Environment, Fisheries and Aquaculture Science
seedcorn program.




\section*{References}

\bibliography{mybibfile}

\end{document}
